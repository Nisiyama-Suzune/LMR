In terms of the implication graph, two literals belong to the same strongly connected component whenever there exist chains of implications from one literal to the other and vice versa. Therefore, the two literals must have the same value in any satisfying assignment to the given 2-satisfiability instance. In particular, if a variable and its negation both belong to the same strongly connected component, the instance cannot be satisfied, because it is impossible to assign both of these literals the same value. As Aspvall et al. showed, this is a necessary and sufficient condition: a 2-CNF formula is satisfiable if and only if there is no variable that belongs to the same strongly connected component as its negation.

This immediately leads to a linear time algorithm for testing satisfiability of 2-CNF formulae: simply perform a strong connectivity analysis on the implication graph and check that each variable and its negation belong to different components. However, as Aspvall et al. also showed, it also leads to a linear time algorithm for finding a satisfying assignment, when one exists. Their algorithm performs the following steps:

Construct the implication graph of the instance, and find its strongly connected components using any of the known linear-time algorithms for strong connectivity analysis.

Check whether any strongly connected component contains both a variable and its negation. If so, report that the instance is not satisfiable and halt.

Construct the condensation of the implication graph, a smaller graph that has one vertex for each strongly connected component, and an edge from component i to component j whenever the implication graph contains an edge uv such that u belongs to component i and v belongs to component j. The condensation is automatically a directed acyclic graph and, like the implication graph from which it was formed, it is skew-symmetric.

Topologically order the vertices of the condensation. In practice this may be efficiently achieved as a side effect of the previous step, as components are generated by Kosaraju's algorithm in topological order and by Tarjan's algorithm in reverse topological order.

For each component in the reverse topological order, if its variables do not already have truth assignments, set all the literals in the component to be true. This also causes all of the literals in the complementary component to be set to false.

